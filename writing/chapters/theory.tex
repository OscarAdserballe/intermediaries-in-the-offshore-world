\chapter{Theory}
\label{chap:theory}

\section{A Note on Philosophy of Science and Methodological Approach}
\label{sec:2_0}

In line with perspectives advocating for methodological pluralism and the use of qualitative insights for broader theory development (e.g., George \& Bennett, 2005), this thesis leverages Harrington's findings for concept formation and hypothesis generation - effectively using her work as a theory-building step. Her work helps define the "intermediary" phenomenon and suggests the importance of factors like trust and expertise, which likely underpin the network structures we observe. This thesis then seeks to assess the generalizability and structural manifestations of these insights across a large dataset, moving from micro-level understanding to macro/meso-level patterns. The objective, therefore, explicitly shifts from \textit{verstehen} to identifying and analyzing recurrent structural patterns within the network as revealed by the ICIJ Offshore Leaks Database \textit{and assuming we can generalise these structures} (more on that later under the methods section).

\section{Conceptual foundations}
\label{sec:2_1}

This section outlines the necessary conceptual foundations that precede the concrete propositions asserted later in the thesis. These concepts presented here as being analytically requisite for the propositions developed in the subsequent section (2.2).

\subsection{Global Wealth Chains and the Role of Intermediaries}
\label{subsec:2_1_1}

To understand the significance of the intermediaries central to this thesis – the professional advisors, lawyers, accountants, and wealth managers operating within the offshore financial system – it is helpful to adopt an analytical framework that explicitly centers their role. The overall motivation for focusing on these actors stems from the "Global Wealth Chains" (GWC) approach.

As articulated by Seabrooke \& Wigan (2014), this approach offers a distinct perspective compared to analyses focused on global value chains. They argue that: "While actors in value chains share an interest in transparency and coordination, those in wealth chains thrive on rendering movements through the chain opaque. Wealth chains hide, obscure and relocate wealth to the extent that they break loose from the location of value creation and heighten inequality." Adopting this GWC lens necessitates an explicit focus on the intermediaries and professionals. These are the actors who develop and deploy the sophisticated financial and legal innovations required to sustain and manage the complex structures used to hold individual wealth offshore, often obscuring its origins and ownership.

Further elaborating on the socio-legal dynamics underpinning these chains, Seabrooke \& Wigan (2022) emphasize the significance of socially constructed legal meaning. They write: "What is significant here is accepted legal assertions,. This happens within interpretative communities, where agreements on legal affordances are secured." The intermediaries operate within these communities, shaping and interpreting the boundaries of legal possibility. Seabrooke \& Wigan (2022) also connect this to broader social valuations, noting that "An important element is that within such communities wealth confers honor, where the accrual and transfer of wealth without productive effort is held in high esteem (Veblen, 1899)." 

Borrowing from the typology proposed in Seabrooke \& Wigan (2022), the networks involving the intermediaries examined in this thesis align closely with their definition of "relational wealth chains."  These are characterized as follows: "Relational wealth chains involve the exchange of complex tacit information, requiring high levels of explicit coordination. Strong trust relationships managed by prestige and status interactions make switching costs high." This description of relational wealth chains, emphasizing tacit knowledge, trust, coordination, and high switching costs due to the personal nature of the relationships, is highly with the ethnographic work of Harrington (2016) and how she outlines the structure and dynamics of the networks between wealth managers and their elite clients. This connection is also drawn by Seabrooke \& Wigan (2022) themselves, who cite Harrington (2015) alongside related work by Beaverstock \& Hall (2016) and de Carvalho \& Seabrooke (2016) as evidence supporting the characteristics of relational wealth chains.

Furthermore, a developing body of literature situated within this GWC approach is examining how these professionals actively shape and navigate existing regulatory landscapes (e.g., Christen, 2021; Christensen \& Seabrooke). This underscores the analytical purchase of the GWC framework for understanding the pivotal role of intermediaries not just as passive facilitators, but as active agents within the offshore system.

\subsection{Weaponised Interdependence}
\label{subsec:2_1_2}

The goal here is to outline the theoretical basis for viewing intermediaries not just as facilitators, but as potential points of leverage or vulnerability within the offshore system, thereby informing regulatory strategies.

A lens for such an analysis is provided by the concept of "weaponised interdependence," as developed by Farrell \& Newman (2019). Their core argument posits that globalization, far from simply flattening the world or diminishing state power, has often created highly specific network topographies. These global networks—whether in finance, technology, or supply chains—are frequently characterized by asymmetric structures. Power, in this view, does not dissipate but rather concentrates at key hubs or 'chokepoints' within these networks. States or actors who control these chokepoints gain significant leverage over others who depend on access to the network, potentially allowing them to 'weaponize' this interdependence for strategic gain.

This logic of weaponised interdependence has been applied directly to the domain of global tax policy by Christensen (2024). He argues that states have often failed to fully harness the potential regulatory power they could wield by strategically targeting chokepoints within the networks facilitating tax avoidance and evasion. Among the key institutions Christensen (2024) identifies as potential chokepoints relevant to global tax policy are precisely the expert intermediaries – the lawyers, accountants, wealth managers, and corporate service providers – who are central to this thesis. Their specialized knowledge and gatekeeping function position them as critical nodes whose disruption could have widespread effects.

This perspective aligns with and provides a theoretical underpinning for findings across various studies highlighting the importance and potential vulnerability of the intermediary supply-side. Research emphasizing the role of intermediaries (e.g., Harrington 2016; Alstadsæter et al. 2019) implicitly points to their structural significance. For instance, Harrington's (2016) focus on trust-based relationships suggests that disrupting these specific intermediary nodes can create significant friction. Alstadsæter et al.'s (2019) supply-side explanation for high-end evasion similarly underscores the crucial role of these facilitators. More explicitly, recent work analyzing the network structures revealed by leaks, such as Chang et al. (2023), demonstrates the analytical purchase of focusing on these networks. While their specific study examined network structures to understand the effectiveness of sanction regimes against oligarchs, the underlying approach – analyzing network vulnerabilities by focusing on intermediary connections – is directly applicable to the broader question of regulating the offshore system for tax purposes.

All in all, understanding the network structure, particularly the role of intermediaries as potential chokepoints, reinforces the idea that the current state of offshore finance and associated tax evasion is, as Saez \& Zucman (2019) argue in a related context, a continued choice shaped by policy and enforcement priorities, rather than an immutable fact of nature. 

\subsection{Network Theory as a Lens for Understanding Illicit networks}
\label{subsec:2_1_3}

To further contextualize the approach taken in this thesis, it is useful to briefly elaborate on how network studies have previously been employed to explore the structure and dynamics of analogous social and economic systems. The application of network analysis provides powerful tools for understanding complex relational patterns, information flows, and vulnerabilities within various types of networks, including those operating in clandestine or illicit domains.

The foundational work in social network analysis, such as Granovetter's (1973) seminal paper on the "strength of weak ties," laid the groundwork for understanding how network structures facilitate crucial processes like information diffusion and resource access. While initially focused on phenomena like job searching, these core insights into how different types of ties (strong vs. weak) and different network positions (e.g., bridges) shape outcomes have proven broadly applicable. Understanding the topology of connections is essential for identifying critical links, potential weaknesses, and influential actors within any network system. This foundational understanding extends to the analysis of illicit networks, where mapping relationships can reveal operational structures and vulnerabilities.

One of the prominent examples demonstrating the application of network analysis to understand illicit operations is the work of Morselli (2009). By examining specific cases, such as the CAVIAR network involved in cross-border drug smuggling, Morselli illustrates how network science concepts (like centrality measures, brokerage roles, and structural holes) can be used to dissect the organizational structure of criminal enterprises. Such analyses move beyond individual actors to understand the relational patterns that enable the illicit activity, potentially identifying key players or structural weaknesses that could be targeted for disruption.

More directly relevant to the subject matter and data source of this thesis, recent studies have begun applying network analysis to the large-scale datasets released by the ICIJ. Chang et al. (2023), for instance, utilized network methodologies on ICIJ data to specifically examine the effectiveness of sanction regimes against oligarchs, analyzing how their embeddedness within offshore networks influenced outcomes. Similarly, related work by the same authors ("Complex Systems of Secrecy," Chang et al. 2023) employed network perspectives to explore patterns related to the types of offshore instruments demanded by elites, linking structural features to strategic choices. These studies exemplify how network analysis can yield substantive insights from the complex relational data contained within the ICIJ leaks, demonstrating its utility for exploring the offshore financial system.

The general principles and analytical techniques drawn upon in such studies are well-established within the broader field of network science, with standard references like Newman's (2010/2018) textbook providing comprehensive overviews of the underlying theory and methodologies. While this thesis may focus more on synthesis and proposition-building informed by network concepts rather than complex quantitative modeling, drawing upon this established body of work provides a robust conceptual and methodological grounding for analyzing the structure and significance of intermediary networks in offshore finance.

\subsection{A Typology of Intermediaries and Their Role}
\label{subsec:2_1_4}

To proceed with an analysis centered on the supply-side, it is essential to clarify conceptually what exactly is meant by an "intermediary" within the context of offshore finance. These actors play diverse roles in facilitating the creation, maintenance, and utilization of offshore structures. While specific studies, such as Harrington (2016), provide deep insights into the practices of particular intermediary types like wealth managers, a broader classification is useful for systematic analysis.

These are all what Hoang (2022) would call the "small spiders", the "High net worth indivduals" rather than the "Ultra-High net worth individiduals" sitting at the top of the food chain. Anything uncovered, in this respect is extremely limited, because they are able to further obfuscate their position.

For this purpose, this thesis builds upon the typology developed in a 2017 EU report examining the role of advisors and intermediaries as revealed in the Panama Papers. This framework, grounded in empirical observation of a major leak, categorizes intermediaries based on their primary area of expertise and function within the offshore ecosystem. Adopting this typology serves a dual purpose: it provides conceptual clarity for the subsequent discussion and offers a practical schema for efforts to classify the varied intermediary actors identified within the ICIJ dataset, thereby enriching the data for structural analysis.

Based on the EU (2017) framework, we can distinguish the following core types of intermediaries:
\begin{itemize}[leftmargin=*]
    \item \textbf{Tax Experts:} These intermediaries focus primarily on the tax implications of offshore structures. Their core function involves advising clients on tax planning strategies to minimize liabilities (potentially crossing into evasion) and ensuring compliance through the preparation of necessary tax documentation across relevant jurisdictions. This group can include accountants, auditors, and specialized tax advisors, whose advice may vary in aggressiveness.
    
    \item \textbf{Legal Experts:} This category encompasses professionals providing expertise on the legal design, establishment, and enforcement of offshore structures. Key activities include structuring entities to navigate or exploit laws in multiple jurisdictions, handling incorporation (often via licensed entities), drafting legal documents, arranging nominee services, and providing formal legal opinions or representation. This group includes regulated lawyers, who often have exclusive rights for certain actions like court representation, and potentially notaries involved in document formalization.
    
    \item \textbf{Administrators:} The primary role of administrators is the ongoing operational maintenance and financial record-keeping of offshore entities. This includes preparing financial accounts, potentially handling tax returns (overlapping with Tax Experts), managing day-to-day administrative tasks, and sometimes auditing accounts (though auditors require independence). Accountants often fall into this category, focusing on financial recording and reporting.
    
    \item \textbf{Investment Advisors:} Distinct from those setting up the structure, investment advisors focus on managing the assets held within the offshore entity. Their core function is to develop strategies for wealth preservation or growth using the financial instruments (or other assets like property, art, etc.) owned by the offshore structure. Their role is centered on asset management rather than the legal or tax architecture itself.
\end{itemize}
This typology provides a decent conceptual grounding for analyzing the distinct roles and potential influence of different supply-side actors within the offshore financial network.

\subsection{Secrecy Strategies: Financial Instruments and Legal Innovations}
\label{subsec:2_1_5}

Goal: Understanding the different financial instruments they use and how they can be innovated on, and used for different purposes. (Mainly Lafitte, 2024; Chang et al. 2023)


Most important type, Bearer instruments:

Harrington (2016) writes of Bearer instruments as follows:
*In addition, a few offshore jurisdictions allow the use of “bearer shares,” which are a way of issuing corporate stock without specifying a particular owner. Rather, the owner of a bearer share is literally whoever happens to be holding the stock certificate at any moment in time. This provides strong privacy protections, because as long as one does not have the shares in hand, one can say truthfully under oath, “I do not own that firm.” And if any officers of the firm are ever questioned about its ownership, they can also truthfully say, “I don’t know who owns the company, because bearer shares were issued.” In other words, bearer shares make it impossible to know who owns a company, and that makes it impossible to assign legal responsibility for any taxes, fines, or debts the company incurs.*


\section{Propositions}
\label{sec:2_2}

PLACEHOLDER






