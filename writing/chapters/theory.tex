\chapter{Theory}
\label{chap:theory}

Four main sections are presented in this chapter to situate the theoretical background and existing literature of intermediaries. As a structural device, each subsection answers a specific "question" about intermediaries - the "who", "what", "where" of intermediaries. 

First, situating them within Global Wealth Chains, situating the "Where" they are located, understood as in which structures. The micro-sociological accounts of primarily Harrington (2016) and Hoang (2022) will be viewed as answering "Who" these intermediaries are. Then, a section on the "What" - going more into their concrete activities as well as a typology of the functional specialisation of intermediaries from De Groer (2017). Lastly, the final section details how ICIJ data has previously been used to study the offshore financial system to generally situate the extent to which it can be validly used in preface of the next chapter on the overall empirical strategy.

Throughout the thesis, I will leverage the same broad definition of intermediaries ICIJ uses. They define them as follows:

\begin{quote}
    [...] [A]n Intermediary refers to a node representing individuals or firms that facilitate the creation and management of offshore entities. These intermediaries include lawyers, accountants, and service providers who assist in setting up and maintaining offshore companies, trusts, or other legal entities. (ICIJ, n. d.)
\end{quote}


\section{Where do Intermediaries fit in? - Global Wealth Chains}

To understand the operational domain of intermediaries - the "where" of their activities - the concept of Global Wealth Chains (GWCs) offers a clear framework in which to place their activities. Seabrooke and Wigan (2017, p. 2) define GWCs as "transacted forms of capital operating multi-jurisdictionally for the purposes of wealth creation and protection." This framework moves beyond static geographical location to situate intermediaries within the dynamic, networked, and often opaque processes through which global wealth is managed, moved, and shielded. The "where" of intermediaries, then, is understood as their position within these complex, multi-jurisdictional chains that exploit the disjuncture between global capital mobility and territorially-bound fiscal and regulatory systems (Seabrooke \& Wigan, 2017).

Intermediaries are active components within these chains and fundamental to shaping them. Professionals are the engineers and architects within the GWCs, acting as the micro-level agents who 1) connect disparate legal and financial systems, thereby enabling the macro-level structures of offshore finance, and 2) often are among the primary forces shapign the very regulation that they are subject to (Christensen et al., 2022; Christensen, 2020; Harrington, 2016). They are "located" at the critical junctures where different national regulations meet, exploiting the seams and gaps between them for the benefit of their clients (Seabrooke \& Wigan, 2014; Christensen et al., 2022). This involves structuring entities, managing information flows, and ensuring (or circumventing) compliance across various legal territories, a process that often relies on cultivating opacity rather than transparency (Seabrooke \& Wigan, 2017). While the term "offshore" explicitly evokes images of remote island nations, most intermediary activity in GWCs occurs within the major onshore financial centers themselves that Stausholm and Garcia-Bernardo (2024) identify as "tax coordination centers." The expertise driving GWCs is, therefore, often concentrated in the very OECD countries that ostensibly seek to regulate them. 

As a role, they are only set to get bigger, benefitting from the increasing complexity of international regulation. As Bustos et al. (2023) suggest, new regulatory measures can create more business for specialized intermediaries like wealth managers or transfer pricing experts, who are then paid to navigate or even engineer pathways through these new rules. When new OECD transfer pricing regulations were implemented in Chile to counteract transfer pricing misuse for the sake of tax avoidance, the number of transfer pricing experts at the Big Four in the country increased from 8 to 95; needless to say, the regulation had no significant effect (Bustos et al., 2023).

In essence, intermediaries are found "where" the legal, financial, and regulatory complexities of the globalized economy are most acute, and "where" expert knowledge can be leveraged to facilitate the multi-jurisdictional logic of wealth creation and protection inherent in Global Wealth Chains.

\section{Who are "Intermediaries"?} 

To delineate "who" these intermediaries are, we turn to micro-sociological accounts that detail their professional identities in thick, ethnographic accounts allowing to get a sense of the nature of their relationships. Brooke Harrington (2016) vividly encapsulates the role through the evocative image of Mr. Tulkinghorn, the lawyer from Charles Dickens' Bleak House. Specializing in trusts and estates, Tulkinghorn is the quintessential keeper of secrets, the one who knows everything about everyone. As Harrington (2016, p. 1) quotes, "He is surrounded by a mysterious halo of family confidences, of which he is known to be the silent depository." The intermediary to their clients often serves as a guardian of sensitive information, whose core value lies in discretion and intimate knowledge, often cultivated through "relationships of long and uncertain duration, usually measured in lives" (Harrington, 2016, p. 79), particularly in the case of wealth managers who are the primary focus of her ethnography.

While Harrington’s deep dive centers on wealth managers, the fundamental characteristics she uncovers-the paramount importance of trust, discretion, and sophisticated relational work-resonate across the spectrum of intermediaries crucial to the offshore world. Trust and relation-building is of the quintessential capactiy to these professionals serving this "politically and socially homogenous and autonomous group" of inidividuals (Harrington 2016, p. 81). Whether it be wealth managers, tax advisors or legal experts, these business relations are built on a foundation of trust, confidentiality and personal rapport (see, for example, Neely, 2021; Hoang, 2022; even Shiller, 2012, makes a large point of this type of trust as underlying all financial intermediation). The entire edifice of offshore finance, designed to create and maintain opacity, hinges on such trusted relationships. In these specialized markets, as Hanlon (cited in Harrington, 2016, pp. 14-15) notes, "Reputational capital [is] at the apex of selling complex products in professional markets." Secrecy is the product, and trust is the indispensable service these intermediaries are selling. The case is often, as one source is quoted in Harrington (2016, p. 85), 'No one in my family knows that this structure exists; only you, me and my lawyer know about it.’ 

The question then obviously arises: \textit{who} possesses the capabilities to cultivate and embody such profound trust, particularly in contexts demanding utmost confidentiality and navigating complex, often morally ambiguous, terrains? The literature points overwhelmingly to the power of homophily and shared cultural understanding (Neely, 2021; Ho, 2009). Intermediaries are often those who can successfully traverse the "trust-barrier" (Harrington, 2016) by leveraging cultural and social similarity with their clients. This involves deploying a shared habitus, in Bourdieu's (1977) terms—a system of dispositions, tastes, and ways of being that resonate with the "politically and socially homogenous and autonomous group" of wealthy individuals they seek to serve (Harrington, 2016, p. 81). Hoang’s (2020; 2022) ethnography of "spiderweb capitalism" similarly private equity partners in foreign markets primary job is selling an idea of similarity to investors, and bonding with them through "homoerotic" relations and bonding experiences. It is relational work all the way down.

In sum, the "who" of intermediaries, from a micro-sociological standpoint, is defined by their capacity to embody trust in a deeply personal and culturally resonant manner for each client. They are professionals – often lawyers, accountants, or specialized wealth managers – who possess not only technical expertise but, more critically, the social, cultural, and relational capital required to become indispensable confidants. They are the human interfaces in a system built on opacity and the "silent depositories" of their clients' most sensitive financial affairs, adept at translating global systems into personalized solutions for wealth protection.

\section{What Functions do Intermediaries Have?}

Having established \textit{where} intermediaries are situated within Global Wealth Chains (GWCs) and \textit{who} these actors are in terms of their relational and cultural capital, this section addresses the "what": What specific functions do intermediaries perform, and what tools do they employ to achieve their clients' objectives?  As the architects and engineers of the offshore world, their work relies on the deployment of specific "legal technologies" offered by various jurisdictions, instruments that are fundamental to facilitating the multi-jurisdictional arbitrage and opacity characteristic of GWCs (Seabrooke \& Wigan, 2017; Christensen et al., 2022; Lafitte, 2024). 

\subsection{Legal Technologies used by Intermediaries}

Jurisdictions compete on the legal technologies they offer, and the possibilities they offer intermediaries and beneficiaries in terms of the structures they can incorporate (Lafitte, 2024). Lafitte (2024) expanding on the view of states selling sovereignty as in the seminal paper by Palan (2002), constructs a historical dataset citing a range of legal handbooks to construct which specific legal technologies different sovereignties provide. In microeconomic parlance, the "selling" of sovereignty and their tax laws is the extensive margin (whether they offer it at all) and the "legal technologies" are the intensive margin (the extent to which they offer it) (Palan, 2002; Lafitte, 2024). The most important ones include trusts enabling the separation of legal and beneficial ownership, bearer shares allowing for anonymous ownership, and nominee services providing a layer of separation between the ultimate beneficial owner (UBO) and the legal entity (Lafitte, 2024; Harrington, 2016). 

\subsection{Functional Specialisation of Intermediaries}

While the "who" section, drawing on Harrington (2016), highlighted the relational work of wealth managers, the offshore ecosystem involves a broader cast of professionals, each contributing distinct functions. De Groen (2017), in his analysis following the Panama Papers leak, provides a useful four-fold typology of intermediaries based on their primary area of expertise and function. This typology helps to disaggregate the "what" of intermediary work and understand how different specialists contribute to the GWC by utilizing the aforementioned legal technologies:

\begin{itemize}[leftmargin=*]
    \item \textbf{Tax Experts:} These intermediaries focus primarily on the tax implications of offshore structures. Their core function involves advising clients on tax planning strategies to minimize liabilities (potentially crossing into evasion) and ensuring compliance through the preparation of necessary tax documentation across relevant jurisdictions. This group can include accountants, auditors, and specialized tax advisors, whose advice may vary in aggressiveness.
    
    \item \textbf{Legal Experts:} This category encompasses professionals providing expertise on the legal design, establishment, and enforcement of offshore structures. Key activities include structuring entities to navigate or exploit laws in multiple jurisdictions, handling incorporation (often via licensed entities), drafting legal documents, arranging nominee services, and providing formal legal opinions or representation. This group includes regulated lawyers, who often have exclusive rights for certain actions like court representation, and potentially notaries involved in document formalization.
    
    \item \textbf{Administrators:} The primary role of administrators is the ongoing operational maintenance and financial record-keeping of offshore entities. This includes preparing financial accounts, potentially handling tax returns (overlapping with Tax Experts), managing day-to-day administrative tasks, and sometimes auditing accounts. Accountants often fall into this category, focusing on financial recording and reporting.
    
    \item \textbf{Investment Advisors:} Distinct from those setting up the structure, investment advisors focus on managing the assets held within the offshore entity. Their core function is to develop strategies for wealth preservation or growth using the financial instruments (or other assets like property, art, etc.) owned by the offshore structure. Their role is centered on asset management rather than the legal or tax architecture itself.

\end{itemize}

\section{How has ICIJ Data been used to Study the Offshore Financial System?}

Researching the clandestine world of offshore finance, and the intermediaries who enable it, presents inherent challenges due to the system's defining characteristic of secrecy (Chang et al. 2023a). The International Consortium of Investigative Journalists (ICIJ) Offshore Leaks Database, however, offers an unparalleled entry-point into this world, given its scale and granularity. As Kejriwal \& Dang (2020, p. 3) note, the database's strength lies in its mapping of an otherwise secret global system:

\begin{quote}
    "[...] [P]recisely because the collection maps out a global system, the Panama Papers also present us with a golden opportunity to study the flow of information between firms, individuals and intermediaries. [...] Studying the structural properties of this complex system using applied networks science has the potential to reveal interesting trends about how such systems operate across geographies and economies."
\end{quote}

The ICIJ database, however, is not a comprehensive or randomly sampled representation of the entire offshore world. It is a compilation of data from specific leaks, each with its own origins and potential biases. For instance, a significant portion of the data originates from particular service providers like Mossack Fonseca (Panama Papers) or Appleby (Paradise Papers). Consequently, observed patterns in clientele, jurisdictions, and service types may, to some extent, reflect the operational focus and market position of these specific firms rather than the offshore industry in its entirety (De Groen, 2017). 

A core challenge in studying offshore finance is identifying the Ultimate Beneficial Owners (UBOs), who often employ sophisticated techniques to obscure their connection to assets - even obfuscation robust to leaks. Hoang (2022), in her ethnography, notes an example of High Net Worth Individuals (HNWI) appearing as "fall guys" named in the ICIJ papers, while the Ultra-HNWI (UNHWI) remains obfuscated; there are layers to secrecy, and those exposed in the leaks often only represent the comparatively more visible part. While this opacity surrounding UBOs is a significant limitation for some research questions especially those focusing on the beneficiaries from the offshore using this dataset, it is less prohibitive for the study of intermediaries. Intermediaries, by their very nature as facilitators and often as the direct points of contact between clients and service providers, and are frequently explicitly named within the leaked data; they \textit{are} the fall guys in Hoang's (2022) sense.

Even with those limitations in mind, its pragmatic value to get an inside look in this otherwise ever-so reclusive world, is illustrated by the increasing use of the data to gauge general patterns in other academic research. Studies have employed it to gauge propensities for offshore use across income distributions (see, for example, Alstadsæter et al., 2019; Londoño-Vélez \& Ávila-Mahecha, 2021), to explore relationships between offshore structures and political contexts (Chang et al., 2023a; Chang et al., 2023b), and to analyze the network structure of offshore finance (Kejriwal \& Dang, 2020). This thesis will proceed with similar caution, focusing on identifying general patterns and relational dynamics concerning intermediaries rather than numerical estimates.



