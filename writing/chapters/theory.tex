\chapter{Theory}
\label{chap:theory}

\section{A Note on Philosophy of Science and Methodological Approach}
\label{sec:2_0}

This study investigates the structure and vulnerabilities of offshore intermediary networks using quantitative network analysis. The methodological approach adopts a logic common in \textbf{mixed-methods research}, specifically leveraging insights from qualitative, in-depth studies to inform large-scale quantitative analysis (cf. Creswell \& Plano Clark, 2017; Tashakkori \& Teddlie, 2010). While not conducting new ethnographic fieldwork, the research builds conceptually upon the rich insights generated by such work, particularly Harrington's (2016) seminal study of wealth managers. Harrington's ethnographic approach, aimed at achieving deep contextual understanding (\textit{verstehen}), provides an invaluable foundation for identifying key actors, understanding their functional significance, and generating plausible hypotheses about the mechanisms shaping the offshore financial ecosystem.

In line with perspectives advocating for methodological pluralism and the use of qualitative insights for broader theory development (e.g., \textbf{George \& Bennett, 2005}), this thesis leverages Harrington's findings for \textbf{concept formation and hypothesis generation}. Her work helps define the "intermediary" phenomenon and suggests the importance of factors like trust and expertise, which likely underpin the network structures we observe. This study then seeks to assess the \textbf{generalizability} and \textbf{structural manifestations} of these insights across a large dataset, moving from micro-level understanding to macro/meso-level patterns. The objective shifts from \textit{verstehen} to identifying and analyzing recurrent structural patterns within the network revealed by the ICIJ Offshore Leaks Database, aiming for what George \& Bennett might term \textbf{contingent generalizations} – patterns potentially dependent on specific contexts (like governance).

Methodologically, this involves integrating insights across different traditions, a process supported by several strands of contemporary social science methodology:

\begin{itemize}[leftmargin=*]
    \item \textbf{Ontology: Compatibility within Constructivism and Relationalism.} Operating within a constructivist framework compatible with Harrington's focus, this research assumes that social structures, like intermediary networks, arise from the interactions, roles, and shared understandings of actors. This aligns with \textbf{relational sociology} approaches that view structures as emergent from social relations (Emirbayer, 1997). While ethnography explores the micro-dynamics, this study focuses on the \textbf{emergent structural properties} amenable to quantitative network analysis. The ontological stance posits that these socially constructed networks possess measurable characteristics and real-world consequences, bridging micro-level interactions with macro-level patterns (cf. Hedström, 2005, on mechanism-based explanations which often link micro and macro).

    \item \textbf{Epistemology: From Interpretive Understanding to Pattern Identification via Abduction.} This study explicitly distinguishes its epistemological goal from \textit{verstehen}. Following a logic akin to \textbf{abductive reasoning} (Peirce; Timmermans \& Tavory, 2012), Harrington's rich observations suggest plausible theoretical models (e.g., intermediaries acting as specialized hubs). This research then seeks to evaluate the empirical adequacy and generalizability of these models by testing for \textbf{observable, structural patterns} in large-N data. Knowledge is generated through the systematic analysis of network topology, centrality, and correlation with external factors. This aligns with the broader scientific goal, emphasized by scholars like \textbf{George \& Bennett (2005)}, of using rigorous methods (whether qualitative or quantitative) appropriate for the specific type of inference being made – in this case, descriptive and correlational inference about network structures across many cases.

    \item \textbf{Axiology: From Thick Description to Generalizable Systemic Analysis.} The primary value pursued remains the generation of \textbf{generalizable knowledge} about system-level properties relevant to understanding systemic vulnerabilities and the influence of context (e.g., governance). This focus on identifying broader patterns and their correlates, potentially informing policy or comparative theory, is a widely recognized goal in social science research, complementing the indispensable value of deep contextual understanding provided by ethnography.
\end{itemize}

In essence, this study employs a \textbf{theory-building strategy} that starts with concepts refined through qualitative research (Harrington) and proceeds to quantitative testing and generalization across a large dataset (ICIJ). This reflects established practices in mixed-methods research and aligns with calls (e.g., George \& Bennett, 2005) for integrating diverse methodological strengths to build more robust and comprehensive social scientific knowledge. The approach explicitly acknowledges the different forms of knowledge generated by ethnographic versus large-N quantitative analysis but frames them as complementary rather than incompatible.

\section{Conceptual foundations}
\label{sec:2_1}
Covering the necessary conceptual foundations preceding the concrete propositions asserted here in the thesis. These are analytically requisite to understand the propositions in the following section, 2.2.

\subsection{Role of Intermediaries as Professionals (Seabrooke, Wigan, Christensen, Tsingou, Harrington) - the whole CBS cartel}
\label{subsec:2_1_1}
\begin{itemize}[leftmargin=*]
    \item Goal: background of why we should think professionals ot be important.
\end{itemize}

\subsection{Weaponised Interdependence for importance of the supply-side as chokepoints (Farrell \& Newman, 2019, Christensen 2024, Zucman 2019) and Network Vulnerability through Intermediaries (Harrington 2016, Alstadsæter et al. 2019, Chang et al. 2023)}
\label{subsec:2_1_2}
Goal: Why it's relevant to specifically understand these networks for regulatory

\subsection{Network Theory as a Lens for Understanding Illicit networks (Chang et. al 2023, Newman textbook, Newman 2003 overview paper, Barabási, A.-L., \& Albert, R. (1999))}
\label{subsec:2_1_3}
Goal: briefly elaborate how network studies have been used to explore this general type of network


\subsection{The specific roles of "intermediaries" and a Typology of Intermediaries (EU 2017 paper, Harrington 2016)}
\label{subsec:2_1_4}
Goal: Clarify conceptually what exactly we mean by an "intermediary"

Harrington (2016) focusing on Wealth Managers as the specific intermediary.
FIGURE OUT WHAT THEIR RELATION OT OTHER INTERMEDIARIES ARE. ARE THEY COORDINATING TO REACH OUT TO THE OTHERS? A WHOLE DIFFERENT CLIENTELE?

Using EU (2017) paper on the role for advisors and intermediaries in the Panama Papers as the one to build on. Typology of Advisors in Offshore Schemes (Based on Primary Expertise). This is also the one we'll be following to start enriching the ICIJ dataset, trying to classify the different intermediaries into thsi schema.

\textbf{Tax Experts}:

Core Function: Primarily involved in advising on tax positions (planning) and preparing tax returns (compliance) related to offshore structures.
Key Activities: Designing strategies to legally lower effective tax rates (tax avoidance), potentially crossing into illegal evasion. Helping UBOs navigate tax laws across jurisdictions. Preparing necessary tax documentation.
Distinctions: While sometimes involved in tax disputes, formal representation in court is often exclusive to lawyers. Tax advice itself is often not a protected or specifically regulated profession, though advisors might be part of regulated firms (like accounting firms). Can include internal experts, accountants, auditors, or other external preparers. The aggressiveness of tax planning advice can vary.

\textbf{Legal Experts}:

Core Function: Providing expertise on the legal aspects of designing, creating, maintaining, and enforcing offshore structures across relevant jurisdictions.
Key Activities: Structuring entities to comply with (or exploit) laws in onshore and offshore jurisdictions. Handling incorporation and registration (where legally permitted, e.g., via licensed trust companies). Drafting legal statutes and documents. Arranging or acting as nominee directors/shareholders. Representing clients in legal disputes/litigation. Providing legal opinion letters.
Sub-Types Mentioned:
Lawyers: Typically regulated professionals, advise, prepare documents, represent clients, often bound by confidentiality.
Notaries: Can play a role in drafting/recording founding documents, managing share transfers, and record-keeping, often seen as performing a public service.
Distinctions: Require deep knowledge of multiple legal systems. Certain activities (like court representation or specific registration tasks) are often restricted to licensed lawyers.

\textbf{Administrators}:

Core Function: Primarily involved in the ongoing maintenance and financial record-keeping of offshore structures.
Key Activities: Preparing financial accounts. Potentially preparing tax returns (overlapping with Tax Experts). Examining/auditing financial accounts (if required or requested). Providing opinion letters on financial constructions. Managing day-to-day administrative tasks.
Sub-Types Mentioned:
Accountants: Focus on preparing financial accounts and often tax returns; may advise on tax planning. Often not a protected profession, but may belong to professional bodies.
Auditors: Independently examine financial accounts. Assess accuracy and control processes. Increasingly subject to independence rules restricting non-audit services. Review tax provisions within financial statements.
Distinctions: Focused on financial recording, reporting, and verification. Auditors specifically require independence from the client. Offshore entities in key Panama Papers jurisdictions often lacked mandatory audit requirements.

\textbf{Investment Advisors}:

Core Function: Advising UBOs on how to manage and invest the assets held within the offshore structures.
Key Activities: Developing strategies for wealth preservation or growth using assets (e.g., shares, debt securities, derivatives) held by the offshore entity. Organizing wealth held in various forms (financial assets, property, yachts, art etc.).
Distinctions: Role is specifically focused on the management of assets within the structure, rather than the creation or legal/tax compliance of the structure itself. Less critical if structures primarily hold non-financial assets like personal luxury goods

\subsection{Different financial instruments and legal innovations - secrecy strategies (Chang et al. 2023, Lafitte, 2024)}
\label{subsec:2_1_5}
Goal: Understanding the different financial instruments they use and how they can be innovated on, and used for different purposes. E.g.

\section{Propositions}
\label{sec:2_2}

\textbf{Proposition 1: Offshore Intermediary Networks Exhibit Scale-Free Centrality and Targeted Vulnerability.}

\begin{itemize}[leftmargin=*]
    \item \textbf{Core Idea:} The network connecting clients to the intermediaries who facilitate offshore structures is characterized by a high degree of centrality, where a few "hub" intermediaries serve a disproportionate number of clients, making the overall network susceptible to targeted disruption aimed at these key players.
    \item \textbf{Building on:} Network science principles (Barabási \& Albert, 1999), empirical findings on offshore networks (Chang et al. 2023 - Complex Systems), supply-side vulnerability arguments (Alstadsæter et al. 2019), and sociological insights on trust concentration (Harrington, 2016).
    \item \textbf{Contribution:} While Chang et al. (2023) identified scale-free properties in specific contexts (e.g., sanctioned oligarchs), this proposition aims to \textit{systematically quantify} these structural properties across the broader Offshore Leaks Database, explicitly link intermediary centrality to network vulnerability, and test if specific intermediary \textit{types} dominate these central positions.
    \item \textbf{Sub-propositions:}
    \begin{itemize}[leftmargin=\parindent]
        \item \textbf{1a. Power-Law Distribution of Intermediary Connectivity:} The distribution of the number of clients served per intermediary will follow a power law, indicating the presence of highly connected "hub" intermediaries. \textit{Contribution:} Confirms and quantifies the scale-free nature specifically for the \textit{intermediary} nodes across the comprehensive ICIJ dataset.
        \item \textbf{1b. Hub Intermediary Dominance and Network Fragility:} The overall connectivity and integrity of the client-intermediary network are disproportionately dependent on a small number of high-degree intermediary hubs. Simulated or observed removal of these top intermediaries (ranked by centrality metrics) will lead to significantly greater network fragmentation compared to random node removal. \textit{Contribution:} Explicitly tests the vulnerability implication of the scale-free structure by focusing on intermediary removal, linking network science to Alstadsæter et al.'s (2019) supply-side control argument.
        \item \textbf{1c. Functional Specialization of Hub Intermediaries:} Certain \textit{types} of intermediaries (as defined in 2.1.2 - e.g., Legal Experts, Administrators) will be significantly over-represented among the high-centrality network hubs, suggesting that specific functions are critical bottlenecks in the offshore facilitation process. \textit{Contribution:} Adds a functional dimension to the structural analysis of hubs, investigating \textit{which kinds} of intermediaries become central.
    \end{itemize}
\end{itemize}

\textbf{Proposition 2: Intermediary Network Structures and Vulnerabilities Vary Systematically with Client Home Country Governance.}
\begin{itemize}[leftmargin=*]
    \item \textbf{Core Idea:} The way clients connect to intermediaries, the resulting network structure, and the vulnerability of key intermediaries are not uniform globally but are shaped by the political and institutional environment (rule of law, regime type, corruption) of the clients' home countries.
    \item \textbf{Building on:} Chang et al. (2023 - Secrecy Strategies), Alstadsæter et al. (2019 - inequality/evasion links), comparative institutionalism, and sociological work on trust in different contexts (Harrington, 2016).
    \item \textbf{Contribution:} Extends Chang et al.'s work on secrecy \textit{strategies} by systematically examining how client home country governance shapes the \textit{measurable network structure} centered on intermediaries (e.g., intermediary centrality distributions, clustering, specialization) and the \textit{differential vulnerability} of these intermediary networks, using comprehensive governance indicators (WJP, VDEM) across the full dataset.
    \item \textbf{Sub-propositions:}
    \begin{itemize}[leftmargin=\parindent]
        \item \textbf{2a. Governance-Dependent Intermediary Network Topologies:} Key intermediary-centric network metrics (e.g., parameters of the intermediary degree distribution, average intermediary clustering coefficient, network density around intermediaries) will differ significantly and predictably based on the governance indicators (WJP Rule of Law, V-Dem Regime Type, TI CPI) of the client cohorts' home countries. \textit{Contribution:} Quantifies the link between macro-level governance and the micro-level structure of intermediary networks.
        \item \textbf{2b. Contextual Variation in Intermediary Hub Fragility:} The "super-fragility" observed by Chang et al. (2023) will manifest as heightened network fragmentation upon removal of top \textit{intermediary hubs} specifically for client cohorts from countries with weaker rule of law or more autocratic regimes, potentially reflecting concentrated trust patterns or higher reliance on specific "gatekeepers" in such contexts. \textit{Contribution:} Tests the fragility hypothesis specifically on intermediary hubs and links it systematically to governance indicators.
        \item \textbf{2c. Intermediary Specialization Based on Client Governance Profile:} Intermediaries will exhibit measurable specialization, tending to serve clusters of clients predominantly from countries with similar governance profiles (e.g., intermediaries specializing in clients from high-corruption countries vs. those specializing in clients from strong rule-of-law countries). \textit{Contribution:} Moves beyond geographic specialization to test for intermediary specialization based on the \textit{political-institutional} context of their clientele.
    \end{itemize}
\end{itemize}

\textbf{Proposition 3: The Functional Roles and Network Positions of Intermediary Types Vary with Client Governance Context.}
\begin{itemize}[leftmargin=*]
    \item \textbf{Core Idea:} The \textit{types} of intermediaries (Legal, Tax, Admin, Investment) that are most prominent, central, or interconnected within the network depend on the governance context of their clients, reflecting varying demands for specific forms of expertise (e.g., legal protection vs. sophisticated tax planning vs. basic administration).
    \item \textbf{Building on:} The intermediary typology (EU 2017; Harrington 2016), qualitative insights on intermediary roles, and the logic linking governance to secrecy needs (Chang et al. 2023).
    \textbf{Contribution:} This proposition quantitatively tests the relationship between client governance context and the \textit{functional composition} of the intermediary network. It moves beyond simply identifying types to analyze their \textit{relative prominence, centrality, and interplay} within the network structure as shaped by client origin.
    \item \textbf{Sub-propositions:}
    \begin{itemize}[leftmargin=\parindent]
        \item \textbf{3a. Governance-Driven Prominence of Intermediary Functions:} The relative share of network activity (e.g., number of connections, aggregate centrality) accounted for by different intermediary \textit{types} (Legal, Tax, Admin, Investment) will vary significantly and predictably based on the governance indicators of the client cohorts' home countries. \textit{Contribution:} Quantifies how client context shapes the demand for specific intermediary functions within the network.
        \item \textbf{3b. Contextual Determination of Critical Intermediary Hub Types:} The \textit{type} of intermediary most likely to occupy a high-centrality hub position (as identified in Prop 1c) will differ based on client home country governance. For instance, Legal Experts may be more central hubs for clients from weak rule-of-law states, while Tax Experts may be hubs for clients from high-tax/strong-enforcement states. \textit{Contribution:} Integrates network structure (hub status) with intermediary function (type) and client context (governance).
        \item \textbf{3c. Varying Intermediary Ecosystems by Client Context:} The patterns of co-occurrence or collaboration \textit{between different types} of intermediaries serving the same client will vary based on the client's home country governance. For example, clients from complex regulatory environments might exhibit denser connections involving multiple specialist intermediary types. \textit{Contribution:} Explores how governance shapes the functional ecosystem and potential division of labor among different intermediary types.
    \end{itemize}
\end{itemize}

\textbf{Proposition 4: Intermediary Network Dynamics are Driven by Supply-Side Factors and Regulatory Shocks.}
\begin{itemize}[leftmargin=*]
    \item \textbf{Core Idea:} Changes in the offshore landscape, such as the introduction of new legal vehicles by tax havens or major international regulatory initiatives (like CRS), significantly impact the structure and activity within the intermediary network, influencing which intermediaries are active, their centrality, and how they connect to clients.
    \item \textbf{Building on:} Laffitte (2024) on legal innovations, Alstadsæter et al. (2019) on detection probability/costs, Christensen (2024) and Farrell \& Newman (2019) on regulatory shocks/weaponized interdependence, and O'Donovan et al. (2019) on market reactions to transparency shocks.
    \textbf{Contribution:} While prior work studied the \textit{impact} of these factors, this proposition specifically analyzes their effect on the \textit{intermediary network structure}. It aims to quantify how supply-side changes reshape intermediary activity, centrality, specialization, and overall network topology, potentially using event-study or difference-in-differences approaches applied to network metrics over time.
    \item \textbf{Sub-propositions:}
    \begin{itemize}[leftmargin=\parindent]
        \item \textbf{4a. Intermediary Response to Legal Innovation:} Following Laffitte (2024), the introduction of significant new legal vehicles in offshore jurisdictions will be associated with measurable changes in the activity levels, client base composition, or network centrality of \textit{intermediaries} specializing in or operating from those jurisdictions. \textit{Contribution:} Links macro-level legal changes (Laffitte) to observable shifts in micro-level intermediary network roles.
        \item \textbf{4b. Regulatory Shocks Reshape Intermediary Network Topology:} Major regulatory shocks, particularly the phased implementation of the Common Reporting Standard (CRS), will lead to detectable shifts in the intermediary network structure, such as changes in overall network density, shifts in the centrality ranking of intermediaries, potential exit of certain intermediary types, or altered geographic patterns of intermediary-client links. \textit{Contribution:} Applies event analysis logic (O'Donovan et al.) to track the evolution of the \textit{intermediary network itself} in response to major regulatory interventions (Christensen, Alstadsæter).
        \item \textbf{4c. Heterogeneous Intermediary Reactions to Supply Shocks:} The impact of legal innovations (4a) and regulatory shocks (4b) on intermediaries and their network position will vary significantly depending on the \textit{type} of intermediary (e.g., Tax Experts vs. Administrators might react differently to CRS) and the \textit{governance context} of the clients they primarily serve (linking back to Prop 2 \& 3). \textit{Contribution:} Examines differential effects, recognizing that intermediaries are not a monolithic group and serve diverse clienteles facing different pressures.
    \end{itemize}
\end{itemize}
