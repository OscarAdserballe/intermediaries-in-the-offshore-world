\chapter{Discussion}
\label{chap:discussion}

\section{Three Propositions}

Three core propositions developed

Not in themselves particularly novel - none of them come as a particular surprise - but viewed through a set of new empirical data.
\subsection{Proposition 1: Specialisation of Intermediaries}

Functional Specialisation: Even though there's bound to be measurement error with the approach taken here, yet still significance for at least two distinct sets of roles. Taking outset from the typology of De Groen (2017). 

First set of roles:
\begin{itemize}
  \item Administrators
  \item Legal Experts
\end{itemize}

Second set:
\begin{itemize}
  \item Tax Experts
  \item Investment Advisors
\end{itemize}

\begin{itemize}
\item Both the significance of the degree distribution only between those two sets,
\item likewise entropy measures and bearer instrument share clearly visually distinct and significant across those two groups, but not very much within
\end{itemize}

Correlate with operational scale (degree of connectivity) and the diversity of their service portfolios (jurisdictional and legal-tech entropy). Personalised wealth management versus the mass provision of legal services.

Geographical Specialisation: Distinct development of preferential client corridors linking specific client origination countries to favored offshore jurisdictions, sometimes driven by historical ties, linguistic affinity, or specialized demand. As we'll see in proposition 3, still universal hubs that connect all these corridors.

\begin{itemize}
\item Country heatmaps,
\item Distribution of degree and country connections
\end{itemize}

\subsection{Proposition 2: Duality of Intermediary Focus - Local Anchors, Global Reach}

        Intermediaries often exhibit a primary client concentration within their own operational countries, likely leveraging local networks and trust (Granovetter, 1973; Stausholm, 2024; Harrington, 2016; Hoang, 2022). However, their core value proposition and a key driver of their specialization lies in their capacity to connect these local clients to a diversified global offshore architecture.
\begin{itemize}
\item Heatmaps showing intermediaries in HKG, GBR, USA etc., serving a notable portion of clients from their home country.
\item Lower client country entropy compared to jurisdiction entropy, suggesting more concentration in client origin.
\end{itemize}

\subsection{Proposition 3: Structural Centrality of Microstates in Intermediation Network}
A core set of Offshore Financial Centers (OFCs), that in heavy part are those infamous microstates. Highly central in it.

        Interesting to note, that this may also be due to their offering of versatile ``legal technologies'' like ``Dual-Purpose'' vehicles as for example is proposed in Laffitte (2024). Empirical confirmation that they form the structural backbone of the global offshore network, and are countries that intermediaries from all countries whose residents they do business and jurisdiction they incorporate. Critical hubs and bridges in chains of intermediation, facilitating complex offshore strategies regardless of client or intermediary home country.
\begin{itemize}
\item High centrality (betweenness, eigenvector) of jurisdictions like VGB, BHS, PAN, CYM, HKG in your co-service and co-usage networks.
\item Dominance of ``Dual-Purpose'' legal technologies in the central core of the jurisdiction co-usage network.
\item High lift values between key OFCs in association analysis.
\end{itemize}

\section{Implications for Regulation}

The potential to escape the multi-level games; most intermediaries serving officers in their own countries. Reclaiming lost tax revenue, can be done by targeting local citizens (though still the problem of changing citizenship).

Functional specialisation very much seeming to correlate with the degree: Layered due diligence or liability regimes could be interesting.

Affirming the centrality of microstates in the offshore network, and the potential for targeted regulation. Switzerland and Luxembourg as less active tax havens as case study.

And not least, affirming the importance of intermediaries.

\section{Limitations}

Where to begin. PLACEHOLDER

\section{Future Research}

\begin{itemize}
\item A lot within the current dataset that could be done - like extending a lot of the analyses. More importantly, the whole temporal dimension is currently left out; are the patterns persistent? As seen, network here has entities over a very large time frame, countless of them that are not active anymore.
\end{itemize}



