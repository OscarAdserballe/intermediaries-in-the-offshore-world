\chapter{Discussion}
\label{chap:discussion}

\section{Two Propositions}
\label{sec:two_propositions}

The empirical journey through the dataset allows for the development of two central propositions regarding the nature and operation of intermediaries in the offshore financial system. While the underlying ideas may not be entirely novel in themselves—indeed, they echo themes present in the broader literature—this thesis offers a fresh perspective by examining them through the lens of a large-scale, aggregated dataset of leaked information, providing quantitative grounding for often qualitatively observed phenomena.

\subsection{Proposition 1: Intermediaries Exhibit Significant Functional and Geographical Specialisation}
\label{subsec:prop1_specialisation}

The data strongly suggests that the intermediary landscape is not homogenous. Instead, intermediaries appear to develop distinct specialisations, both in the types of services they offer (functional specialisation) and in the geographical markets they serve or utilise (geographical specialisation).

\subsubsection*{Functional Specialisation}
The analysis of intermediary classifications, taking its outset from the typology developed by DeGroen2017, reveals a clear divergence in operational characteristics. Despite the inherent challenges and potential for measurement error in classifying intermediaries based on publicly available information, statistically significant differences emerge, particularly when grouping intermediaries into two broader sets of roles. The first set, comprising \textbf{Administrators} and \textbf{Legal Experts}, tends to be associated with higher-volume activities. This is evidenced by their generally higher degree distributions (as seen in Figure \ref{fig:specialisation_cdf_degrees}), suggesting a larger client load. Furthermore, these types often exhibit greater diversity in their operational metrics, such as the legal technologies employed and the jurisdictions utilised for incorporation (Figure \ref{fig:specialisation_average_entropy_bearer}), consistent with a role in facilitating a wide array of entity formations and management tasks. This aligns with the notion of these professionals providing the infrastructural backbone for the offshore system, handling the mechanics of entity creation and maintenance, potentially akin to the "Captive" or "Hierarchy" Global Wealth Chains (GWCs) described by SeabrookeWigan2017, where professional firms or in-house departments manage complex, often high-volume, structures.

The second set, consisting of \textbf{Tax Experts} and \textbf{Investment Advisors}, generally displays lower degrees of connectivity. Their operational profiles, as indicated by entropy measures, often show a more focused approach to client countries and incorporation jurisdictions. This pattern is consistent with these professionals offering more bespoke, personalised advice, which inherently limits the scale of their direct client engagement. Such roles might align more with "Relational" GWCs, where trust and tacit knowledge are paramount SeabrookeWigan2017, or reflect the "strategic" professional action described by ChristensenEtAl2022ProfessionalAction, involving tailored solutions rather than mass provision. The lack of significant differentiation in bearer instrument usage across these functional types (Figure \ref{fig:specialisation_average_entropy_bearer}) is a notable finding, suggesting that the deployment of such high-anonymity tools may be driven by specific client demands or niche opportunities rather than being a hallmark of a particular intermediary specialisation within this classified sample.

\subsubsection*{Geographical Specialisation}
Beyond functional roles, intermediaries demonstrate marked geographical specialisation. The heatmaps of client countries and incorporation jurisdictions (Figures \ref{fig:geography_country_heatmaps_top5} to \ref{fig:geography_country_heatmaps_cyprus}) reveal distinct national and regional profiles. Many intermediaries, even those in major global financial centres, show a strong tendency to serve entities linked to their own country of operation or immediate region. This is further underscored by the distribution of countries linked to entities served per intermediary (Figure \ref{fig:geography_distribution_countries_by_intermediary}), where the vast majority of intermediaries, irrespective of their overall client load (log-degree), focus on entities active in only one or two countries. This suggests that scaling often occurs through deeper penetration within existing client geographies rather than broad international expansion of the client base.

This geographical concentration in client sourcing likely reflects the importance of local knowledge, networks, and trust, as highlighted by Granovetter1973, Harrington2016, and Hoang2022. However, this localised client focus is often paired with a strategic selection from a global "market for tax havens" Laffitte2024 when it comes to incorporation jurisdictions. The higher entropy observed for incorporation jurisdictions compared to client countries (Figure \ref{fig:geography_country_level_entropy_distribution}) supports this: intermediaries draw upon a wider, more diverse palette of offshore jurisdictions to structure entities, even for a geographically concentrated client base. This points to a sophisticated understanding and utilisation of the specific "legal technologies" offered by different OFCs. The Cyprus-Russia corridor (Figure \ref{fig:geography_country_heatmaps_cyprus}), for instance, illustrates a highly specific linkage, echoing findings by AlstadsaterEtAl2022 on specialized financial conduits. While distinct client corridors emerge, the empirical analysis also hints at the existence of universal hubs (like the BVI for incorporations) that connect many of these otherwise specialised pathways, a theme that resonates with the network structures discussed by KejriwalDang2020.

\subsection{Proposition 2: Duality of Intermediary Focus - Local Anchors, Global Reach}
\label{subsec:prop2_duality}

Building on the geographical specialisation findings, the second core proposition posits a fundamental duality in intermediary operations: they are often locally anchored in their client acquisition but globally oriented in their service provision. Intermediaries frequently exhibit a primary client concentration within their own operational countries or regions, as evidenced by the heatmaps (Figures \ref{fig:geography_country_heatmaps_top5} to \ref{fig:geography_country_heatmaps_top11_15}) where countries like Hong Kong, the UK, and the USA show intermediaries serving a notable portion of clients from their home country. The significantly lower mean entropy for client countries compared to incorporation jurisdictions (Figure \ref{fig:geography_country_level_entropy_distribution}) further substantiates this local concentration in client origin.

This local anchoring likely leverages established networks, cultural affinity, linguistic ease, and, crucially, trust—a cornerstone of the wealth management relationship as detailed by Harrington2016 and foundational to social and economic interactions as per Granovetter1973. The findings of StausholmGarciaBernardo2024, showing tax advisors clustering in major financial centres often serving domestic or regionally proximate clients, also align with this observation.

However, the core value proposition and a key driver of intermediary specialisation lies in their capacity to connect these local or regional clients to a diversified global offshore architecture. Intermediaries act as bridges, translating local client needs into structures that utilise a global array of offshore jurisdictions, each selected for its specific legal, financial, or secrecy advantages. This "global reach" is what allows them to navigate and exploit the complexities of the international financial and regulatory system, offering solutions that transcend purely domestic options. This duality—local trust and access combined with global operational capability—positions intermediaries as critical nodes in the broader offshore network.

\section{Implications for Regulation}
\label{sec:implications_regulation}

The propositions derived from the empirical analysis carry significant implications for regulatory strategy, suggesting both challenges and potential avenues for more effective oversight of the offshore financial system.

\subsection{Collapsing Multi-Level Games and Targeting Chokepoints}
\label{subsec:collapsing_games_chokepoints}

The "local anchor" aspect of Proposition 2—that many intermediaries serve a significant portion of clients from their own country of operation—offers a potential leverage point for national regulators. If intermediaries are indeed facilitating offshore arrangements for domestic clients, then national authorities may have more direct jurisdiction and visibility than if the entire chain were purely "offshore." This could, in principle, help to "collapse the multi-level games" that often characterize international tax evasion and avoidance, where actors exploit seams between national regulatory systems. By focusing on domestically based intermediaries, states might reclaim a degree of control over activities that ultimately impact their own tax base.

This aligns with the growing academic and policy interest in targeting "enablers" or "chokepoints" within financial networks. As argued by Christensen2024WeaponisedInterdependence through the lens of "weaponised interdependence" FarrellNewman2019, states controlling key nodes in global networks can exert significant leverage. Intermediaries, particularly those identified as highly connected "super-hubs" (Figure \ref{fig:preliminary_powerlaw_fit}) or those specializing in servicing domestic elites, could represent such chokepoints. The work of ChangEtAl2023ComplexSystems on the vulnerability of oligarch networks to the targeting of key intermediaries further supports this notion.

This perspective offers a counter-narrative to the often-cited "Retreat of the State" Susan Strange, as cited in][p. 262], which posits that globalization diminishes state power. While Harrington herself notes the perpetual "cat and mouse" game, scholars like SaezZucman2019 and HearsonChristensen2020 argue that the potential for effective regulation, even unilaterally, has been underestimated. Saez and Zucman, for example, advocate for remedial taxes on profits booked in low-tax jurisdictions, a policy that implicitly targets the outcomes facilitated by intermediaries.

However, the path to effective regulation is fraught with complexities. The findings of BustosEtAl2023, showing that the implementation of OECD transfer pricing standards in Chile did not reduce tax-motivated payments but rather spurred the tax advisory industry, serve as a cautionary tale: tax planners can indeed outpace tax enforcement. Moreover, the global regulatory landscape itself is uneven. The United States, with its FATCA regime, has pursued a unilateral approach that, while impactful, stands apart from the OECD's Common Reporting Standard (CRS). This has led some, like Stiglitz2025, to argue that the US itself has become a significant tax haven, attracting capital seeking to avoid CRS reporting. Such dynamics illustrate that targeting chokepoints requires not only identifying them but also navigating a complex geopolitical environment where major players may have divergent interests.

\subsection{Layered Liability and Due Diligence Regimes}
\label{subsec:layered_liability}

The functional specialisation identified in Proposition 1, particularly the correlation between intermediary type and operational scale (degree), suggests that a one-size-fits-all regulatory approach may be suboptimal. Intermediaries like Administrators and some Legal Experts, who manage a high volume of entities, may pose different types of risks and require different supervisory attention than Tax Experts or Investment Advisors, who typically have fewer, more bespoke client relationships.

This differentiation could inform the design of "layered" due diligence or liability regimes. For instance, intermediaries with very high degrees of connectivity, or those identified as "super-hubs," might be subjected to more stringent ongoing monitoring, enhanced reporting obligations, or even a higher standard of care regarding the activities of the entities they service. Conversely, smaller, more specialised advisory firms might be subject to a different, though still robust, set of expectations tailored to the risks associated with in-depth, personalised advice. The typology by DeGroen2017 could serve as a starting point for developing such risk-based categorizations.

Furthermore, understanding the specific "legal technologies" Laffitte2024 or GWC types SeabrookeWigan2017 that different intermediaries specialize in could allow for more targeted regulatory interventions. For example, intermediaries heavily involved in structures known for high opacity (e.g., complex trusts, or historically, bearer shares) might warrant specific scrutiny, aligning with efforts to enhance beneficial ownership transparency as advocated by scholars like Knobel2020. The challenge lies in designing such layered regimes without creating new loopholes or imposing disproportionate burdens on legitimate activities.

\section{Limitations}
\label{sec:limitations}

This study, while leveraging a uniquely comprehensive dataset, is subject to several limitations that must be acknowledged. Firstly, the ICIJ data, while extensive, represents leaked information from a subset of offshore service providers. It is not a complete census of all offshore activity or all intermediaries. Therefore, the patterns observed, particularly regarding the prevalence of certain jurisdictions or intermediary types, may not be fully generalizable to the entire offshore ecosystem. The reliance on Mossack Fonseca data for some analyses, for example, reflects the practices of one (albeit major) player.

Secondly, the classification of intermediaries into functional types based on publicly available information and algorithmic assistance is inherently subject to measurement error. While efforts were made to ensure accuracy, the complexity and often deliberate opacity of intermediary operations mean that some misclassifications are inevitable. This could influence the statistical significance or magnitude of observed differences between functional types.

Thirdly, the analysis is largely static, providing a snapshot based on the aggregated data. The offshore world is dynamic, with intermediaries and their clients constantly adapting to regulatory changes and market opportunities. A full understanding would require longitudinal data that is exceedingly difficult to obtain in this domain.

Finally, while this thesis explores structural and functional characteristics, it does not delve deeply into the motivations or specific illicit activities that may be facilitated by the observed structures, beyond what can be inferred from patterns and existing literature. Establishing causality between specific offshore structures and illicit outcomes remains a significant challenge.

\section{Future Research}
\label{sec:future_research}

The findings and limitations of this thesis point towards several promising avenues for future research.
A significant extension would be to incorporate a more dynamic, temporal analysis. The ICIJ dataset spans several decades (as seen in Figure \ref{fig:Preliminary_Incorporations_over_Time}), offering the potential to study how the network structure, geographical concentrations, and intermediary specialisations have evolved over time, particularly in response to major regulatory initiatives (e.g., FATCA, CRS) or economic shocks. This could test the resilience and adaptability of the offshore system.

Further refinement of intermediary classification methodologies, perhaps incorporating more sophisticated machine learning techniques or integrating diverse data sources (e.g., professional registers, court documents), could yield more granular and robust insights into functional specialisation. Comparative studies using data from different leaks or other sources (e.g., national beneficial ownership registers where available) could help assess the generalizability of the findings presented here.

Deeper investigation into the "super-hub" intermediaries identified is warranted. Qualitative case studies of these highly connected actors, exploring their business models, client acquisition strategies, and compliance practices, could provide rich context to the quantitative network patterns. Similarly, more detailed analysis of specific "client corridors" between origination countries and offshore jurisdictions could uncover the precise mechanisms (e.g., historical ties, specific legal expertise, diaspora networks) that sustain these links.

Finally, research could more explicitly link the structural features of offshore networks to specific types and volumes of illicit financial flows. While challenging, developing methodologies to estimate the "risk premium" or "vulnerability score" associated with different network configurations or intermediary types could provide valuable intelligence for regulatory and law enforcement agencies. Exploring the impact of emerging technologies, such as crypto-assets and decentralized finance, on the structure and operation of offshore intermediation is also a critical area for future inquiry.
