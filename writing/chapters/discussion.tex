\chapter{Discussion}
\label{chap:discussion}

\section{Two Propositions}
\label{sec:two_propositions}

Two core propositions emerge from the empirical analysis of the offshore intermediary network.

\subsection{Proposition 1: Intermediaries Exhibit Significant Functional and Geographical Specialisation}
\label{subsec:prop1_specialisation}

The data strongly suggests that the intermediary landscape is not homogenous. Instead, intermediaries appear to develop distinct specialisations, both in the types of services they offer (functional specialisation) and in the geographical markets they serve or utilise (geographical specialisation).

\subsubsection*{Functional Specialisation - Commodification and Customisation}

Having established the at least moderate reinforcement of roles akin to the ones suggested by DeGroen (2017), there is a clear divergence in operational characteristics. Statistical significance, at least between the two groupings, did emerge. This does complicate it, but what it means is that within the groups themselves, there was no significance of roles, but across them in all pairs, there was significance.

\textbf{Commodification}: The first set, comprising Administrators and Legal Experts, tends to be associated with higher-volume activities. This is evidenced by their generally higher degree distributions (as seen in Figure \ref{fig:specialisation_cdf_degrees}), suggesting a larger client load. Furthermore, these types often exhibit greater diversity in their operational metrics, such as the legal technologies employed and the jurisdictions utilised for incorporation (Figure \ref{fig:specialisation_average_entropy_bearer}), consistent with a role in facilitating a wide array of entity formations and management tasks. This positioning suggests they occupy nodal positions within Global Wealth Chains, serving as infrastructure providers that enable multi-jurisdictional wealth structures through scale economies (Seabrooke \& Wigan, 2017). Their operational characteristics align with what the professional services literature identifies as "platformization" strategies, where firms leverage standardized processes and technologies to serve high volumes of clients across multiple markets (Løwendahl, 2005; von Nordenflycht, 2010).

\textbf{Customisation}: The second set, consisting of Tax Experts and Investment Advisors, generally displays lower degrees of connectivity. Their operational profiles, as indicated by entropy measures, often show a more focused approach to client countries and incorporation jurisdictions. This pattern suggests these professionals occupy specialized advisory positions within wealth chains, capturing value through knowledge premiums rather than volume (Seabrooke \& Wigan, 2017). This aligns with the professional services literature on "customization" strategies, where expertise-intensive firms prioritize bespoke solutions and close client relationships over scale (Maister, 1993; Empson et al., 2015). It is likely under these categories that Harrington's (2016) wealth managers would fall, rather than implicitly generalizing their characteristics to all intermediaries in the ICIJ data as Chang et al. (2023a, footnote 2) do.

The method employed will very likely have introduced some kind of measurement error. Random measurement error is not an issue in itself apart from the fact that it will simply reduce the power of the, but systematic (non-classic) measurement error is, however, a potential source of bias (e.g. Evdokimov \& Zeleneev, 2023). One situtation where it could plausibly arise is when intermediaries straddle multiple functional categories as inevitably happens in a lot of these cases - there is even a lot of overlap in the underlying typology - and they are systematically classified into one category over another. This is something that should have been verified using a robustness check, but was not. 

\subsection{Proposition 2: Geographical Specialisation - Local Anchors, Global Reach}
\label{subsec:prop2_duality}

Beyond functional roles, intermediaries demonstrate marked geographical specialisation. The major results in support of this are as follows:

\textbf{Local Anchors}: The heatmaps of client countries and incorporation jurisdictions (Figure \ref{fig:geography_country_heatmaps_top5}) point towards distinctly \textbf{regional} profiles. Many intermediaries, even those in major global financial centres, show a strong tendency to serve entities linked to their own country of operation or immediate region. Likewise, even if they serve many entities, they often do so in a very limited number of countries.

\textbf{Global Reach}: At the jurisdiction-level, they are however far more spread out across a larger band of OFCs, not displaying the same kind of regional concentration. This is evidenced by the significantly higher jurisdiction entropy measures (Figure \ref{fig:geography_country_level_entropy_distribution}), the heatmaps providing individual examples and the distribution of jurisdiction linked to entities served per intermediary by their degree revealing a weak correlation in contrast to degree-countries served (Figure \ref{fig:geography_distribution_countries_by_intermediary}).

This geographical concentration in client sourcing likely reflects the importance of local knowledge, networks, and trust, as highlighted by Granovetter1973, Harrington2016, and Hoang2022. However, this localised client focus is often paired with a strategic selection from a global "market for tax havens" Laffitte2024 when it comes to incorporation jurisdictions. Intermediaries draw upon a wider, more diverse palette of offshore jurisdictions to structure entities, even for a geographically concentrated client base. This points to a higher sensitivity of the specific "legal technologies" offered by different OFCs. The Hong Kong-China corridor (Figure \ref{fig:geography_country_heatmaps_cyprus}), for instance, illustrates a highly specific linkage, echoing findings by AlstadsaterEtAl2022 on specialized financial conduits (see also Cyprus-Russia corridor in appendix). While distinct client corridors emerge, the empirical analysis also hints at the existence of universal hubs (like the BVI for incorporations) that connect many of these otherwise specialised pathways (A more rigorous treatment using network analysis and beyond the individual cases leveraged here of this is discussed in the appendix in Proposition 3).

This is the core value proposition of intermediary specialisation: they have the capacity to connect these local or regional clients to a diversified global offshore architecture. Intermediaries act as bridges, translating local client needs into structures in the international financial and regulatory system that transcend purely domestic options. This duality-local trust and access combined with global operational capability-positions intermediaries as critical as well as vulnerable nodes in the broader offshore network.

The results here were quite robust to the different specifications and methods employed, but nevertheless the temporal dimension is the most significant limitation. The analysis is largely static, providing a snapshot based on the aggregated data. The offshore world is dynamic, with intermediaries and their clients constantly adapting to regulatory changes and market opportunities. A full understanding would require longitudinal data that is exceedingly difficult to obtain in this domain.

\section{Implications for Regulation}
\label{sec:implications_regulation}

The propositions derived from the empirical analysis carry significant implications for regulatory strategy, suggesting both challenges and potential avenues for more effective oversight of the offshore financial system. 

\subsection{Collapsing Multi-Level Games: No Wealth Chain is an Island}
\label{subsec:collapsing_games_chokepoints}

Intermediaries become clear potential targets of \textit{national} regulation. Chang et al. (2023a) provide a clear argument for the level of vulnerability in these networks with targeted removal of the key nodes that emerge because of the power-law distribution. This thesis goes further with the "local anchor" and regional specificity attribute of these intermediaries uncovered: they can potentially even be regulated the domestic level instead. Such an approach offers a strategy to "collapse" the often-intractable multi-level games of international tax governance, which, as Rixen's (2008) work illustrates, can become mired in asymmetric prisoner's dilemmas when attempting to address under-taxation globally. If intermediaries are indeed primarily facilitating offshore arrangements for domestic clients, then national authorities possess a direct regulatory entry-point into these chains. There's furthermore the very real risk that these specialisations are only evidence of short-term rather than long-term stickiness. This aspect of the wealth chain clearly anchors them back to the domestic tax base that is otherwise dead in the hold and being sailed offshore. In the words of Archimedes, "give me a lever and a place to stand and I will move the earth."


Indeed, Christensen (2024) identifies the targeting of intermediaries as a key application of "weaponised interdependence," a mechanism to "reinvigorate regulation." He argues this offers an "effective counterweight to mobile capital's structural power" (Christensen, 2024, p. 180) precisely because it exploits asymmetries in the global system. The United States' Foreign Accounts Tax Compliance Act (FATCA) serves as a prime example: it "weaponised" the US's indispensable role in the global financial system, leveraging the fact that "banks in the liberal economic system can no longer survive without access to the US-controlled, dollar-based financial system" (Emmenegger, cited in Christensen, 2024, p. 179) to compel foreign financial institutions into compliance. This thesis suggests that the "local anchor" status of certain intermediaries provides a similar, albeit domestically focused, leverage point. Such an approach reinforces the arguments of scholars like Saez \& Zucman (2019) and Christensen (2020; 2024) that the potential for effective regulation, even unilaterally or at the national level, has been underestimated. For instance, just as Saez and Zucman advocate for remedial taxes on profits booked in low-tax jurisdictions, financial hubs could target their domestic intermediaries with specific regulations designed to curtail their ability to facilitate offshore schemes, potentially crippling key nodes in GWCs. The precise legal mechanisms is outside the scope of the thesis because of the legal intricacies involved, but the overarching point is that it offers individual nations more options to play the regulatory game at the domestic level.

This is not to say the solution is a panacea; unilateral actions have limits, and pose different threats to the global system. While Christensen (2024) highlights The United States, with its FATCA regime, as a highly impactful unilateral approach, they have at the same time been reluctant to engage with broader multilateral regulations like the OECD's Common Reporting Standard (CRS); the result, like Stiglitz (2025) for example comments, is that the US itself has become the most significant tax haven, attracting capital seeking to avoid CRS reporting. This potential threat is also remarked by Farrell \& Newman (2023) that the dynamic aspects introduced by such (mis)uses of the interdependence of the system; the EU, for one, could for example de-lever their reliance if such a mechanism is, say, used by the US too frequently. The original difficulty of coordinating international tax policy remains, although in different form, and with power shifted toward those major powers like the US that can unilaterally impose their will on the rest of the world, in contrast to the micro-states that previously had the asymmetric advantage of "selling" their sovereignty.



\subsection{Layered Liability and Due Diligence Regimes}
\label{subsec:layered_liability}

The functional specialisation identified in Proposition 1, particularly the correlation between intermediary type and operational scale (degree), suggests that a one-size-fits-all regulatory approach may be suboptimal. Intermediaries like Administrators and some Legal Experts, who manage a high volume of entities, may pose different types of risks and require different supervisory attention than Tax Experts or Investment Advisors, who typically have fewer, more bespoke client relationships.

This differentiation could inform the design of "layered" due diligence or liability regimes. For instance, intermediaries with very high degrees of connectivity, or those identified as "super-hubs," might be subjected to more stringent ongoing monitoring, enhanced reporting obligations, or even a higher standard of care regarding the activities of the entities they service. Conversely, smaller, more specialised advisory firms might be subject to a different, though still robust, set of expectations tailored to the risks associated with in-depth, personalised advice. The typology by DeGroen2017 could serve as a starting point for developing such risk-based categorizations.

