\chapter{Introduction}
\label{chap:introduction}

\section{Introduction} 
\label{sec:1_0}

"How Globalization Really Works", Palan et al. (2010) wrote, is tax havens. The global financial landscape is increasingly characterized by intricate cross-border structures, many of which operate within the opaque realm of offshore finance. The overall phenomenon is well-known as well as the individuals and entities involved are common wisdom, but knowledge on the exact workings of it are a lot rougher.

Keynes (apocryphally) quipped "the avoidance of taxes is the only intellectual pursuit that still carries any reward." It is no wonder that a whole industry has evolved around this facilitation of such avoidance as a "race" has emerged between those making the rules, and those putting their intellectual industries to work to circumvent it (Bustos et al. 2022; Slemrod, 2019). This industry is powered by a diverse array of professional intermediaries - lawyers, accountants, trust companies, and financial advisors, for example - who act as the architects and gatekeepers of the offshore world.

This thesis delves into the ecosystem of these offshore intermediaries, leveraging the extensive micro-data revealed by the International Consortium of Investigative Journalists (ICIJ). The central objective is to illuminate the operational logic of these crucial actors by advancing and empirically testing two primary claims: first, that intermediaries exhibit a high degree of geographical specialization, predominantly serving clients from their own home countries and regions; and second, that distinct patterns of functional specialization exist among the different types of intermediaries operating within the offshore system.

The claim to (moderate) originality of this research is twofold. It presents one of the first systematic empirical studies of intermediary specialization within the global offshore system, moving beyond anecdotal evidence or case studies to map broader patterns. Furthermore, it introduces a novel methodological approach, employing an agent-based classification technique to categorize intermediaries based on their online presence. By providing a more granular understanding of how these intermediaries operate, this thesis aims to shed light on the mechanisms that underpin the global offshore economy, offering insights relevant to policymakers, regulators, and scholars seeking to understand and address the challenges posed by financial opacity and cross-border tax avoidance. The "race" between regulators and intermediaries is a losing battle if nothing is known of them.

What follows is a brief overview of the issue of tax evasion and avoidance primarily motivating the research question, a historical account of the offshore world as a "wicked" problem, and a discussion of the intensification of regulation post-2008. This is followed by an overview of the role of intermediaries as targets of regulation, and finally, the research question and a roadmap for the thesis.

\section{Tax Evasion, Tax Avoidance and the Offshore World}
\label{sec:1_1}

The ability of states to fund public goods and services, mitigate inequality, and ensure social mobility hinges critically on systems of taxation; "a nation", Burke (1774) wrote, "is its system of taxation." The ability of a state to collect the taxes it is owed is fundamental to state capacity (Tilly, 1990). The Offshore world has been a significant barrier to this as tax avoidance and evasion schemes have routed through here, challenging public finances worldwide, eroding tax morale and leading to significant revenue losses (Slemrod, 2019). These practices distort economic incentives, generates a class of socially undesirable rents accruing to groups like intermediaries and exacerbate societal inequalities, short-circuiting society's ability to deliberate about optimal taxation regimes, a concern brought to the forefront by scholars such as Piketty, Zucman and Saez (e.g., Piketty, 2014; Saez \& Zucman, 2016). 

The most important section of the income distribution in terms of tax revenues are the wealthiest and those who utilise these offshore structures (Alstadsæter et al., 2019; Londo{\~n}o-V{\'e}lez et al., 2021) Traditional economic models of tax evasion  and empirical detection methods, such as random tax audits, often fail to capture the full extent and sophistication of strategies employed by the wealthiest individuals (Allingham \& Sandmo, 1972; Kleven et al., 2011). As Alstadsæter, Johannesen, and Zucman (2019) demonstrate through analyses of leaked offshore data, tax evasion among the ultra-wealthy is not only more prevalent but also qualitatively different, often involving complex offshore arrangements that render it largely invisible to conventional tax enforcement mechanisms. Indeed, for the wealthiest echelons, the line between aggressive tax avoidance and outright evasion can become blurred, facilitated by access to specialized advice and intricate financial engineering (Christensen, Seabrooke, \& Wigan, 2021). While outright evasion has been significantly impacted by recent regulatory initiatives, falling as much as a factor of three, avoidance still represents a significant issue for the public purse (Alstadsæter et al. 2024).

The pervasive use of offshore financial centres (OFCs) is not limited to individual tax planning; it is also a cornerstone of corporate tax avoidance strategies on a global scale. Multinational corporations (MNCs) employ a range of mechanisms, most notably transfer pricing methods, to allocate profits to low-tax or no-tax jurisdictions, often disconnecting taxable income from the locations of substantive economic activity (Tørsløv, Wier, \& Zucman, 2018). The magnitude of this profit shifting is significant. For instance, Saez and Zucman (2019, p. 77) estimate that approximately 60\% of the foreign profits of US multinationals are booked in low-tax jurisdictions, far outpacing the real economic activity occurring in those locations; Wier \& Zucman (2018) estimate 10\% of global corporate tax revenues is expropriated from the state because of profit shifting; and Zucman (2015) estimates at least 8\% of the world's financial wealth is held offshore.

Non-tax motivated reasons, it should be noted, for employing offshore structures do exist. These can include facilitating cross-border investments and pooling capital for private equity funds to avoid double-taxation, or seeking a legal and political stability not available in an investor's home country (Carter, 2017; Harrington, 2016). Likewise, its users also includes those wishing to protect assets from potential expropriation in unstable political environments, ensuring contractual enforcement through neutral legal systems, or distancing operations from local corruption (Hoang, 2022). Without these structures there would likely be a reduction in investment because of an introduction of superfluous frictions to capital, especially towards developing countries with weaker institutional frameworks (Carter, 2017).

As Harrington (2016, p. 171) observes in the context of wealth management, but remains equally pertinent more generally to the entities incorporated in offshore jurisdictions for both corporate and high-net-worth individual purposes: "Tax avoidance—which gets the lion’s share of headlines whenever wealth management makes the news—is only the tip of the iceberg. The larger objective is defending wealth from the many risks it faces, both from without (in the form of political retribution or creditors) and from within (in the form of divorcing spouses or spendthrift heirs)." Thus, these entities are versatile tools used for a spectrum of objectives including liability shielding, asset protection, and succession planning, in addition to tax considerations.


\section{A Brief History of the Offshore}
\label{sec:1_2}

The contemporary offshore financial system, with its complex web of tax avoidance, evasion, and financial secrecy, is not a recent anomaly but the outcome of a long historical evolution. This section presents the history of tax havens through the historical institutionalist account provided by Rixen (2010), traces the genesis of this system to the sequential manner in which international tax governance developed.

In the early to mid-20th century, the primary challenge confronting states was the issue of international double taxation, where the same income could be taxed by both the country of source and the country of residence of the taxpayer. This significantly hampered cross-border trade and investment. To address this, states engaged in a process of coordination, culminating in a network of bilateral Double Taxation Treaties (DTTs), largely based on model conventions developed by the League of Nations and later the OECD. This initial phase successfully mitigated double taxation by establishing rules for allocating taxing rights, such as the distinction between active business income (primarily taxed at source) and passive income (often taxed at residence), and foundational concepts like "permanent establishment" and the "arm's length principle" for intra-company transactions (Rixen, 2010).

However, the very architecture designed to resolve double taxation inadvertently sowed the seeds for a new, more intractable problem: international tax competition and widespread opportunities for tax avoidance and evasion. As Rixen (2010) argues, the DTA regime, by preserving national tax sovereignty (allowing countries to set their own rates and define key terms within the treaty framework) while facilitating capital mobility, transformed the strategic landscape. Once the risk of double taxation was largely removed, some jurisdictions suddenly stood in a position to gain by offering minimal or zero tax rates and high levels of financial secrecy, thereby attracting mobile capital and corporate profits at the expense of their previous domestic tax base. This shifted the international tax system from a coordination game to an asymmetric prisoner's dilemma. Larger, higher-tax nations faced erosion of their tax bases, while smaller jurisdictions could specialize in offering offshore services, effectively "commercializing their sovereignty" (Palan, 2002; Laffitte, 2024).

\section{The Intensification of Regulation post-2008}
\label{sec:1_3}

The past two decades have witnessed a significant intensification of international regulatory efforts aimed at curbing offshore tax evasion, aggressive tax avoidance, and illicit financial flows. Spurred by the 2008-2009 global financial crisis, a series of high-profile data leaks (such as the Panama, Paradise, and Pandora Papers - serving as the central data for this thesis), and mounting public pressure, major economies and international bodies like the G20, the OECD, and the Financial Action Task Force (FATF) have brought a new wave of regulation.
Key developments include the implementation of Automatic Exchange of Information (AEOI) through the Common Reporting Standard (CRS), heightened scrutiny of beneficial ownership information, and more stringent Anti-Money Laundering and Counter-Terrorist Financing (AML/CFT) obligations, the BEPS (Base Erosion and Profit Shifting) initiative (see, e.g., Carter, 2017; OECD, 2014; De Groen, 2017). This is of a \textit{qualitatively} different caliber compared to the incrementalist regulations of yesteryear hesitant to work outside the bounds of old institutions (Hearson \& Christensen, 2019).  Alstadsæter et al. (2024) note the three-fold decrease in tax evasion, for example; BEPS is a historic 130+-country agreement to, among much else, enforce a minimum corporate tax; Switzerland as a tax haven, despite being historical cornerstone  has been forced to make significant concessions with the aggressive regulations unilaterally imposed by the US under Obama with FATCA (Foreign Account Tax Compliance Act) (Zucman, 2015; Christensen, 2024). These are un-heard of developments, and harbingers of regulative proposals that are a lot less insistent to work within the historical scope of these institutions (Hearson \& Christensen, 2019).

\section{Intermediaries as a Target of Regulation}
\label{sec:1_3}

Professional intermediaries, right about now, seem particurlaly interesting strategic points of intervention because of their role as the human form that is "racing" against the regulators (to borrow the image of Bustos et al. (2022)) and actively countervailing the intended impact of laws and regulations (Christensen, 2024; Chang et al., 2023a). Several compelling reasons underpin why these actors represent particularly salient targets for regulatory action. 

First, the incentives structuring the behavior of intermediaries are arguably much more sensitive to changes in the regulatory or reputational environment than their clients. For these professionals and firms - ranging from large financial institutions and global accounting firms to specialized law practices and corporate service providers - the provision of offshore services is not merely an ancillary option but often a core component of their business model and professional identity (Christensen, Seabrooke, \& Wigan, 2021). Their professional existence and profitability are directly dependent on their continued ability to offer these specific services effectively and discreetly. Consequently, factors that threaten this ability - such as increased regulatory scrutiny, heightened enforcement risk, or significant reputational damage - can have a pronounced impact on their willingness and capacity to supply these services. In contrast, the demand for tax minimization or asset protection among potential clients, driven by factors like high tax rates or a desire for secrecy, can be seen as a relatively persistent force. While demand might fluctuate, the fundamental desire among some wealthy individuals and corporations to reduce tax burdens or shield assets is likely to remain, making demand potentially less elastic to targeted interventions than the specialized supply of enabling services. The threat of reputational and criminal consequences falls asymmetrically on intermediaries.

Second, the structure of the market for offshore services itself points towards the strategic importance of intermediaries. There often exists a many-to-one relationship between clients and intermediaries; that is, a relatively small number of specialized intermediary firms or key professionals service a large number of clients seeking offshore solutions (Stausholm \& Garcia-Bernardo, 2024; Chang et al., 2023a). This concentration means that the intermediary sector represents a point of leverage, or a "chokepoint" in the network of offshore facilitation (Christensen, 2024). Regulatory actions or enforcement efforts focused on these key intermediary players could potentially have a cascading effect, impacting a wide network of clients far more efficiently than attempting to identify and pursue each individual client separately. This structural feature makes the intermediary supply-side particularly vulnerable, and thus relevant, from a regulatory perspective. A granular understanding of how these intermediaries specialize - geographically and functionally, as this thesis explores - can therefore offer further insights into the nature of these chokepoints and thus potentially the design of regulation.

\section{Research Question: How do Intermediaries Specialise in Offshore Networks?}
\label{sec:1_5}

The preceding sections have established the significant scale of tax avoidance and evasion, a brief history of the offshore financial system that enables these practices, the increasing regulatory attention being paid to it, and why the professional intermediaries who facilitate access to this system might constitute good targets. This leads directly to the central research question of this thesis: \textbf{How do intermediaries specialise in offshore financial networks?}

While a growing body of literature addresses various facets of offshore finance, the specific patterns of intermediary specialization have not been explored systematically. Existing ethnographic research (e.g., Harrington, 2016; Hoang, 2022) has provided rich insights into the micro-level interactions, trust dynamics, and operational practices of wealth managers and their elite clients (cf. Neely, 2022). However, the qualitative nature of these studies, while offering depth, inherently limits the ability to generalize findings to broader, systemic patterns of intermediary behavior across the entire offshore ecosystem. Concurrently, a nascent but expanding stream of research has begun to leverage large-scale leaked datasets or developed novel ones to explore structural aspects of the offshore world (e.g., Chang et al., 2023a; 2023b using ICIJ data as well; Stausholm \& Garcia-Bernardo, 2024; Kejriwal \& Dang, 2020). 

This thesis aims to fill this gap by empirically investigating patterns of intermediary specialisation along two principal dimensions: first, the \textit{geographical specialisation}, examining the countries of the clients they serve and jurisdictions they incorporate entities in; and second, the \textit{functional specialisation}, exploring whether there are distinct types of intermediaries that cater to different client segments or specialize in the provision of particular types of offshore structures or services.

\section{Roadmap of the Thesis.}
\label{sec:1_6}

This thesis is structured as follows, moving from theoretical foundations and methodology to empirical analysis and discussion of implications.

\textbf{Chapter \ref{chap:theory}} (Literature Review and Theoretical Framework) establishes the conceptual groundwork for the study. It reviews the extant literature on offshore finance, the evolving role of intermediaries, and theories relevant to understanding their behavior and specialization. This includes an exploration of 'where' intermediaries operate, drawing on  the Global Wealth Chains framework (Seabrooke \& Wigan, 2017); 'who' these actors are, informed by micro-sociological accounts emphasizing trust, relational capital, and professional networks (e.g., Harrington, 2016; Christensen, Seabrooke, \& Wigan, 2021); and 'what' functions they perform through the deployment of specific legal and financial technologies (Lafitte, 2024; De Groen, 2017).

\textbf{Chapter \ref{chap:data_methodology}} (Data and Methodology) details the empirical strategy employed. It introduces the International Consortium of Investigative Journalists (ICIJ) dataset as the primary source of micro-data, discussing its provenance, structure, variables, strengths, and inherent limitations. This chapter will also present the methodological framework developed for classifying intermediaries and the statistical techniques applied for testing these patterns. A brief overview of prior academic usage of ICIJ data will also be provided to contextualize this thesis.

\textbf{Chapter \ref{chap:empirical_analysis}} (Empirical Analysis) forms the empirical core of the thesis. This chapter presents the findings from the investigation into the two primary dimensions of intermediary specialization. It will first detail the results concerning \textit{geographical specialization}, analyzing patterns of intermediary activity in relation to client home countries and the offshore jurisdictions utilized. Subsequently, it will present findings on \textit{functional specialization}, examining whether and how intermediaries focus on particular types of clients or specialize in facilitating specific offshore structures and services. The analysis will demonstrate that intermediaries are often not generalists but exhibit distinct patterns of specialization along multiple dimensions.

\textbf{Chapter \ref{chap:discussion}} (Discussion) interprets the empirical findings within the broader theoretical and policy context. It explores the implications of the identified patterns of intermediary specialization for understanding the offshore financial system, for national capacity in combating tax avoidance and evasion, and for the design of more effective and tailored regulatory enforcement strategies. 

Finally, \textbf{Chapter \ref{chap:conclusion}} (Conclusion) summarizes the main contributions of the thesis, and outlines some avenues for future research in this area. 
