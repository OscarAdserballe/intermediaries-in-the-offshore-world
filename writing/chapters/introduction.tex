\chapter{Introduction \& Motivation}
\label{chap:introduction}

\section{Introduction} 
\label{sec:1_0}

The central claim advanced throughout this thesis concerns the critical relevance of examining supply-side dynamics within the offshore financial system. Specifically, it argues that the role of intermediaries – the professional enablers and facilitators of offshore activity – is an incredibly relevant factor. The function and influence of the supply side – encompassing the specialized intermediaries and the specific services offered by various jurisdictions that actively enable and shape offshore activity – remains comparatively under-explored from an empirical standpoint. Building upon recent scholarship that increasingly highlights these supply dynamics (e.g., Laffitte 2024; Alstadsæter et al. 2019), this thesis seeks to extend and generalize insights from qualitative work, such as Harrington's (2016) study of wealth managers, through a quantitative analysis drawing upon the extensive data revealed by the ICIJ leaks.

Primary literature this is building on (contextualising interest in the topic):
\begin{itemize}
    \item Interest spurred on this by an interest in optimal taxation regimes esp. Saez (2002), and the work of Zucman \& Saez (2019) on the optimal taxation of wealth. 
    \item Overall approach from neoclassical public finance and economics. Lectures from Zucman's overviews of tax evasion and avoidance in the modern economic literature has been the primary source. \url{https://gabriel-zucman.eu/publicecon/}
    \item Niche within Political Sociology through Brooke Harrington (2016)'s book and the method's employed in her ethnography of wealth managers. Likewise the tentative work in Chang et al. (2023a and 2023b) on network structure. However, for the latter, they concentrate more on demand-strategies rather than the more interesting supply-side strategies that are the focus of thisthesis.
\end{itemize}

\section{Tax Avoidance at the Top of the Income Distribution}
\label{sec:1_1}
While considerable progress has arguably been made in curbing outright tax evasion, tax avoidance remains a substantial challenge, a point emphasized by commentators such as Stiglitz (cited in Alstadsæter et al., 2024). It introduces several clear inefficiencies into the economic system, including the generation of a distinct class and socially unoptimal rents accruing to the intermediaries who facilitate such schemes, the potential for poor allocation of resources as investment decisions are distorted by spurious incentives, and, beyond these economic inefficiencies, a range of normative concerns regarding fairness and the integrity of the tax system that inevitably accompany widespread tax avoidance.

A literature that has grown very prominent in the past two decades or so in  A crucial distinction often highlighted is between income and wealth inequality. Income inequality can be somewhat ephemeral in nature; high-earners in one year may retire or experience income fluctuations in the next. Wealth, in contrast, tends to be more permanent, potentially distorting social outcomes over non-transient periods in a more meaningful way. Inordinate wealth accumulation (e.g. Harrington, 2016) has distorted social mobility (as explored in the work of Chetty) and been a key driver of overall inequality trends (e.g. Piketty's main body of work). 

With that said, from a (narrow and purely economic) point of view, whether tax avoidance quantifying is actually bad is unclear, so the normative desirability of it at aggregate is still in question. The precise behavioral effects of tax evasion and avoidance on incentives – such as the incentives to work, save, or invest – is not as clear as, for example, studying the effects of tax incentives on MNCs (where it seems generally negative, e.g. Puerto Rico tax credit study from Serrato, 2018; also Garrett \& Serrato, 2019). A key complicating factor is the role of expectations; an individual's behavior is likely highly dependent on their expectation of being able to successfully evade or avoid taxes in the future. 

\section{Limitations of Traditional Demand-Side Models}
\label{sec:1_2}

Traditionally, tax evasion and avoidance has been studied from the demand-side. The seminal Allingham-Sandmo (1972) good at explaining tax evasion decisions of the vast majority of the income distribution (Alstadsæter et al. 2019) performs poorly at the top of the distribution (ibid.) the Allingham-Sandmo (1972) model, provides a powerful and often empirically supported framework for understanding tax evasion decisions for the majority of taxpayers. This standard model typically portrays evasion as a individual and rational gamble, where individuals weigh the expected benefits of non-compliance against the probability of detection and the severity of potential penalties (see also Yitzaki \& Slemrod). However, under standard assumptions about risk aversion and the structure of penalties and audit probabilities, the model often predicts that wealthier individuals, facing potentially higher stakes and scrutiny, should be less inclined to evade taxes. Yet, empirical evidence, particularly from studies leveraging leaked data (e.g., Alstadsæter et al. 2019), suggests the opposite: offshore tax evasion appears highly concentrated among the ultra-wealthy. The comparative statics do not hold here.

Furthermore, traditional methods for empirically studying tax compliance, such as random audit studies (e.g., Kleven et al. 2011), also face limitations in capturing the full picture of high-end evasion. As highlighted by Alstadsæter et al. (2019), while random audits are invaluable for understanding compliance behavior regarding income streams typically subject to third-party reporting or easily verifiable through standard audits, they often fail to detect the sophisticated, cross-border evasion strategies frequently utilized by the wealthiest segment. Complex offshore structures, shell corporations, and opaque trust arrangements often fall outside the scope of conventional audit procedures, rendering this form of evasion largely invisible to standard demand-side enforcement tools.

This points towards a dynamic of a game of cat and mouse. Demand-side enforcement mechanisms, predicated on detecting and penalizing individual non-compliance, struggle to keep pace with the evolving and increasingly complex strategies developed to obscure wealth and income, often with the assistance of specialized intermediaries. Consequently, relying solely on demand-side models and traditional enforcement metrics provides an incomplete, and potentially misleading, understanding of the phenomenon, especially concerning the significant evasion occurring at the top of the distribution. This underscores the necessity of incorporating supply-side factors and network structures to actually understand these mechanisms enabling tax avoidance at the top of the income distribution.

\section{The Supply-side: Intermediaries as Gatekeepers}
\label{sec:1_3}

To fully grasp the dynamics of offshore tax evasion and avoidance, it is crucial to clarify what constitutes the "supply-side" (used more-so metaphorically than stringently) in this context. Here, the supply-side refers specifically to the ecosystem of professional intermediaries – such as law firms, banks, trust companies, and specialized advisors – as well as the jurisdictions that provide the legal and regulatory frameworks enabling offshore financial activities. The central argument advanced in this thesis, building on insights from models like Alstadsæter et al. (2019) and qualitative work such as Harrington (2016), is that this supply-side dimension is far more relevant to scrutinize than often acknowledged, potentially offering more effective avenues for understanding and potentially curbing offshore practices compared to a sole focus on demand-side factors.

A primary reason for emphasizing the supply side relates to the concept of elasticity. It is argued here that the elasticity of supply of intermediaries is considerably higher, and therefore potentially more responsive to policy interventions, compared to the elasticity of demand from clients seeking offshore services. Several factors underpin this view:

First, the incentives structuring the behavior of intermediaries are arguably much more sensitive to changes in the regulatory or reputational environment. For these professionals and firms, the provision of offshore services is not merely an option but often a core component of their business model and career trajectory. Their professional existence and profitability are directly dependent on their continued ability to offer these specific services effectively and discreetly. Consequently, factors that threaten this ability – such as increased regulatory scrutiny, heightened enforcement risk, or significant reputational damage – can have a pronounced impact on their willingness and capacity to supply these services. In contrast, the demand for tax minimization or evasion among potential clients, driven by factors like high tax rates or a desire for secrecy, can be seen as a relatively persistent force. While demand might fluctuate, the fundamental desire among some wealthy individuals and corporations to reduce tax burdens is likely to remain, making demand potentially less elastic to targeted interventions than the specialized supply of enabling services.

Second, the micro-sociological account provided by Harrington (2016) offers compelling reasons why intermediaries are so central. Her ethnographic work illuminates the deeply personal, trust-based relationships that often form between wealth managers and their elite clients. These relationships, built over time and predicated on discretion and expertise, are difficult to replace. Clients rely heavily on their chosen intermediaries not just for technical execution but also for navigating the complexities and risks of the offshore world. The non-substitutable nature of these trust-based relationships means that disrupting the intermediary side can significantly impact clients' access to and ability to maintain offshore structures, further highlighting the critical role of the supply-side actors.

Third, the structure of the market itself points towards the strategic importance of intermediaries. There often exists a many-to-one relationship between clients and intermediaries; that is, a relatively small number of specialized intermediary firms or key professionals service a large number of clients seeking offshore solutions. This concentration means that the intermediary sector represents a point of leverage. Regulatory actions or enforcement efforts focused on these key intermediary players could potentially have a cascading effect, impacting a wide network of clients far more efficiently than attempting to identify and pursue each individual client separately. This structural feature makes the intermediary supply-side particularly vulnerable, and thus relevant, from a regulatory perspective.


\section{Research Gap: Understanding the \textit{Network Structure} to Inform Intermediary Regulation}
\label{sec:1_4}

Considerable research, particularly micro-sociological accounts like Harrington's (2016) ethnography, provides rich insights into the dyadic relationships, motivations, and practices of individual wealth managers and their clients. Ethnography, as a methodology, certainly offers a powerful means of accessing and understanding micro-level dynamics that can illuminate macro-level phenomena or "megatrends,"; of "entering in" an otherwise abstract metanarrative (cf. Neely, 2021; Also Chung 2018(check up; misremeber?)) However, generalizing from these detailed qualitative studies to broader systemic patterns has not really been done.

A nascent thread of literature has begun to explore these structural aspects, often spurred by the availability of large-scale leaked data. Work such as Chang et al. (2023), alongside policy-oriented research emerging from bodies like the EU following disclosures such as the Panama Papers (e.g., research from 2017), represents initial steps in this direction. However, this line of inquiry remains limited thus far, often focusing on specific subsets of countries or actors. The analysis of the network structures inherent in the offshore world is still in a highly exploratory phase. Consequently, the potential held within detailed micro-data sources, such as the ICIJ leaks which map connections between entities, officers, and intermediaries on a vast scale, remains largely underexplored in terms of systematic structural analysis.

The work by Chang et al. (2023) on "Secrecy Strategies" provides a pertinent example. While their primary focus was on analyzing the demand strategies employed by global elites, their findings crucially demonstrate that these strategies are shaped by, and interact with, the supply landscape – the available intermediaries, jurisdictions, and the institutional context of the elites' home countries. Their research, therefore, implicitly highlights the importance of the supply structure by showing how it influences demand patterns, effectively linking the two sides of the market through observable strategic choices.

This points towards the specific research gap addressed herein: the need for a more systematic understanding of the network structure of the supply-side itself. While we have compelling accounts of individual intermediary roles and incentives, a comprehensive picture of how these intermediaries connect to each other, to different types of clients, across various jurisdictions, and through specific service offerings – essentially, the topology of the intermediary network – is lacking. Understanding this structure is potentially crucial for designing more effective regulation targeting these key players.

Therefore, the goal within this thesis is to contribute to bridging this gap, primarily through synthesis and systematization. Drawing upon the existing literature, including the rich ethnographic accounts, the aim here is not necessarily to conduct a novel quantitative network analysis but rather to attempt to codify some of the more loosely defined observations about intermediaries and their roles. By viewing these observations through the conceptual lens of network structures and positions, the objective is to formulate more general propositions regarding intermediary behavior, influence, and potential vulnerabilities within the broader offshore system. 

\section{RQ: What role do offshore intermediaries play in networks of high-end tax avoidance?}
\label{sec:1_5}

\section{Roadmap of the Thesis.}
\label{sec:1_6}

Having gone through what motivates the pursuit of this question and situate this thesis, will proceed to the bulk of the paper. First, outline the key concepts and theories I will draw on, then moving on to outline the key propositions this paper will seek to set forth about the role of intermediaries. Then, a brief section will cover the data sources. 
