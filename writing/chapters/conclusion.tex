\chapter{Conclusion}
\label{chap:conclusion}

This thesis covered an exploration of the often-shadowy world of offshore finance, not by chasing the beneficiaries, but by focusing on its architects: the intermediaries. We sought to understand how these crucial enablers specialize, asking \textit{how} they carve out their niches in a system that thrives on complexity and opacity. 

The core contention - that intermediaries exhibit significant geographical and functional specialization - finds support in the analysis. Geographically, many intermediaries are "local anchors with a global reach," deeply embedded in specific national or regional client markets, yet access the global "supermarket" of offshore jurisdictions and their varied "legal technologies." This duality is key to their value proposition. Functionally, the data sketches distinct profiles: the high-volume "commoditizers" like Administrators and Legal Experts, building broad infrastructures of entities, stand in contrast to the more bespoke "customizers" like Tax Experts and Investment Advisors, who offer tailored, often lower-volume, strategic counsel. They point to different operational logics, different scales of connectivity, and potentially different vulnerabilities.

I argue that the implications for those engaged in the "race" against tax avoidance are considerable. If intermediaries are indeed locally anchored, national regulators may possess more direct leverage than often thought, offering a route to bypass the multi-level game of international tax governance. Furthermore, functional differentiation allows for better reasoned regulatory strategies - perhaps layered liability or tailored due diligence as proposed - rather than a "one-size-fits-all" approach.

This thesis has so far scratched the surface, and many interesting avenues remain. Of more theoretical interest, the typology from De Groen (2017) has largely been taken for granted, but how would a bottom-up typology (e.g. using techniques like hierarchical clustering) compare? Here, regime has been one of the lenses taken, but what about, for example, cultural dimensions or geographical distance functions to more firmly accept/reject the "local anchor" proposition? Exploring Microstates in this respect is a particularly interesting case, given their lack of "local" client bases. Finally there's the more empirical side, where lots of interesting questions also arise. The Hong-Kong-China and Cyprus-Russia corridor has briefly been discussed and how they arise in the data, but this could in principle be explored algorithmically to find new corridors using tools like Association Analysis. Much interesting work remains and lots lay out there to be there discovered.

